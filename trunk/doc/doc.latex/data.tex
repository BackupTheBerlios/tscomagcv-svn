%
% data.tex
% Copyright (c) Markus Kohm, 2009
%
% This program is free software; you can redistribute it and/or modify
% it under the terms of the GNU General Public License as published by
% the Free Software Foundation; version 2 of the License.
%
% This program is distributed in the hope that it will be useful,
% but WITHOUT ANY WARRANTY; without even the implied warranty of
% MERCHANTABILITY or FITNESS FOR A PARTICULAR PURPOSE.  See the
% GNU General Public License for more details.
%
% You should have received a copy of the GNU General Public License
% along with this program; if not, write to the Free Software
% Foundation, Inc., 59 Temple Place, Suite 330, Boston, MA  02111-1307  USA
%

\documentclass{scrartcl}
\usepackage[utf8]{inputenc}
\usepackage[T1]{fontenc}
\usepackage{lmodern}
\usepackage[english]{babel}
\usepackage{xspace}
\usepackage{svn}\SVN $Id$
\usepackage{hyperref}

\newcommand*{\Comag}{Comag PVR2/100CI\xspace}

\begin{document}
\title{Data Formats of the Comag PVR2/100CI\footnote{aka Silvercrest SL80/2
    100CI}\,\footnote{This is: ``\SVNId''.}}
\author{Markus Kohm}
\SVNdate $Date$
\maketitle
\begin{abstract}
  The sat receiver Comag PVR2/100CI or Silvercrest SL80/2 100CI has an
  internal HDD and may record to this an play from it. It can also handle
  external storage devices via USB. Recordings are managed to special named
  directories with three management files \texttt{meta.dat}, \texttt{rec.cp},
  \texttt{rec.bm} and several MPEG transport stream files, names
  \texttt{rec.ts}, \texttt{rec.01}, \texttt{rec.02} \dots
  Only special TS files can be used by the receiver.
\end{abstract}

\tableofcontents

\section{The HDD-Organization}
\label{sec:hdd-organization}

The \Comag reads only the first VFAT partition of the HDD or storage
device. This partition type should be FAT32~LBA (0xC) and the filesystem has
to be VFAT. Nevertheless the \Comag does not show long filenames but trunks
all filenames to eight characters.

Records should be done to the root directory and are organized in directories
named \texttt{rec\_\textit{DDDD}} where \texttt{\textit{DDDD}} is a four digit
decimal number. We call these directories \emph{Records}. See
\autoref{sec:records} for more information about.

Note: Sometimes external USB storages may not be recognized. Sometimes it
helps to unplug an replug the USB cable.

\section{The \texttt{rec\_\textit{DDDD}} Directories}
\label{sec:records}

A \emph{Record} is a directory named \texttt{rec\_\textit{XXXX}} with
\texttt{\textit{DDDD}} is a four digit decimal number. Only files at
directories with this naming scheme are recognized by the \Comag to be
playable.

Each Record contains three management files and one or more files with the
transport stream. The management files are:
\begin{labeling}[:]{\texttt{meta.dat}}
\item [\texttt{meta.dat}] This information will be shown in the display and
  the overview but not in the info box or while cutting. It is
  essential. Without this information the record does not work. See
  \autoref{sec:meta.dat-file} for more information about.
\item [\texttt{rec.bm}] This information may be regenerated by the \Comag, if
  it is not there. See \autoref{sec:rec.bm-file} for more information about.
\item [\texttt{rec.cp}] This information may be regenerated by the \Comag, if
  it is not there. See \autoref{sec:rec.cp-file} for more information about.
\end{labeling}

The transport stream is saved at files \texttt{rec.ts}, \texttt{rec.01},
\texttt{rec.02} \dots\@\ Each with a maximum file size of 2.0\,GB\footnote{The
  exact maximum file size seems to be $2\,147\,471\,360\,\textrm{Bytes} =
  11\,422\,720 \cdot 188\,\textrm{Bytes}$. And the first file, \texttt{rec.ts}
  seems to be even smaller.}. See \autoref{sec:rec.ts-file} for more
information about.

\section{The \texttt{meta.dat} File}
\label{sec:meta.dat-file}

\textsc{To be continued}

\section{The \texttt{rec.ts}, \texttt{rec.01}, \texttt{rec.02} \dots Files}
\label{sec:rec.ts-file}

\textsc{To be continued}

\section{The \texttt{rec.bm} File}
\label{sec:rec.bm-file}

\textsc{To be continued}

\section{The \texttt{rec.cp} File}
\label{sec:rec.cp-file}

\textsc{To be continued}

\end{document}

%%% Local Variables: 
%%% mode: latex
%%% mode: reftex
%%% TeX-master: t
%%% End: 
